%        File: 0.tex
%     Created: сб апр 11 02:00  2020 M
% Last Change: сб апр 11 02:00  2020 M
%
\documentclass[algebra,a5paper]{pum}
\listnumber{0}
\date{11.04.20}
\classname{8-Д}
\lesson{Правила}
\geometry{bottom=0cm}

\begin{document}
%Уважаемые учащиеся группы 8-Д!

%По результатам двухнедельного адаптационного этапа обучения в дистанционном режиме были сформулированы следующие правила проведения занятий с использованием двух онлайн сервисов

\subsubsection*{Правила работы в \href{https://zoom.us}{Zoom}-конференции}
\begin{enumerate}[nosep]
  \item При входе в Zoom-конференцию ученик обязан отправить личное сообщение организатору конференции со словом <<я>>.
  \item После того как в чате будет написано слово <<вопрос>>, а сам вопрос будет озвучен учителем или показан на экране, ученик обязан отправить личное сообщение организатору с ответом до озвучивания правильного ответа учителем, который сопровождается написанием слова <<ответ>> в чате.
  \item Ученик обязан написать личное сообщение <<вышел>> организатору в чат при покидании рабочего учебного места, а по возвращении -- сообщение <<вернулся>>.
  \item Ученик может нажать на кнопку <<Поднять руку>> для того, чтобы задать вопрос. Ученик самостоятельно включает звук для того, чтобы озвучить свой вопрос. После получения ответа на вопрос ученик выключает звук. Также допускается задать вопрос без поднятой руки, не перебивая друг друга, если вопрос захотят озвучить два и более учеников.
  \item \textcolor{red}{Online-конференция начинается с началом урока по расписанию.} Урок длится 45 минут, из которых 40 минут дается на online-конференциюи 5 минут на подготовку фотографий проделанной работы для Google Класс. 
\end{enumerate}

\vspace{-0.3cm}
\subsubsection*{Правила работы в \href{https://classroom.google.com/}{Google Класс}}
\begin{enumerate}[nosep]
  \item Ученик обязан заходить в Google Класс перед уроком/практикумом, которые начинаются по расписанию, и открывать соответствующее задание.
  \item Ученик сдает работу в виде фотографий своих записей в тетради в последовательности заданных номеров. На фотографии не должно быть лишних полей, не содержащих записей, а также частей стола и т.д. Качество фотографии должно обеспечивать читабельность рукописного текста, а ориентация фотографии соответствовать правильной ориентации текста для чтения. Допускается фотографировать лист тетради частями.
  \item Любое задание должно быть сдано вовремя. После назначенного срока сдачи учитель проверяет работы согласно указанным критериям и оценивает несданные \textcolor{red}{(не нажата кнопка <<Сдать>>)} работы нулевым баллом.
  \item Ученик обязан просмотреть возвращенную работу и внимательно изучить комментарии. Возникший вопрос отправляет ответным комментарием. Пояснение к полученной отметке можно получить в <<критериях оценки>>.
  \item Во внеурочное время ученик обязан зайти в Google Класс, открыть домашнее задание и выполнить его.
\end{enumerate}

%\subsubsection*{FAQ}
\begin{pumbox}{Ответы на часто задаваемые вопросы:}
  \begin{enumerate}[nosep]
    \item {\bf Оценки в ЭЖД}: выставляются в конце учебной недели. 
    \item {\bf Минимальная оценка за работу}: работа не сдана или не прикреплено ни одного файла.
    \item {\bf Отсутствие на уроке}: ученик не написал слово <<я>> в чате конференции при входе.
    \item {\bf Не удалось прикрепить файлы}: вопрос к интернет-провайдеру или в \href{https://support.google.com/edu/classroom/?hl=ru#topic=6020277}{службу поддержки} Google Класс.
  \end{enumerate}
\end{pumbox}

%\begin{pumbox}{Изменения правил}
%  Учитель оставляет за собой право вносить дополнения в правила для улучшения занятий и объективности оценивания знаний учащихся.
%\end{pumbox}
\end{document}
