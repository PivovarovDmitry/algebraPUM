%        File: 18.tex
%     Created: вт апр 28 10:00  2020 M
% Last Change: вт апр 28 10:00  2020 M
%
\documentclass[algebra,a4paper]{pum}
\listnumber{18}
\date{28.04.20}
\classname{8-Д}
\lesson{практикум}
\begin{document}
\begin{pumbox}{Обобщенный метод интервалов: алгоритм}
  \begin{enumerate}[label=\arabic*.]
    \item Перенести все слагаемые в левую часть неравенства.
    \item Преобразовать левую часть неравенства к дробному выражению.
    \item Разложить числитель и знаменатель на произведения линейных и квадратных многочленов.
    \item Вынести коэффициент при старшей степени каждого сомножителя за скобку.
    \item Разделить неравенство на произведение вынесенных числовых сомножителей.
    \item Расставить свободные члены линейных многочленов по возрастанию на числовой прямой, обозначив их кружками.
    \item Если неравенство нестрогое, то заштриховать круги, соответствующие свободным членам в линейных многочленах числителя.
    \item Расставить между отмеченными точками чередующиеся знаки справа налево, начиная с плюса. Если линейных многочлен стоит в четной степени, то проходя через точку, соответсвующую его свободному члену, \textcolor{red}{не чередуем знак}.
    \item Записать объединение интервалов одного знака, который определяем из неравенства, и жирных точек.
  \end{enumerate}
\end{pumbox}

\begin{exercises}
  \begin{multicols}{2}
    
\begin{question}
  Решить обобщенным методом интервалов неравенства с произведением двух линейных многочленов:
\begin{multicols}{2}
\begin{enumerate}
  \item $(x-2)(x-3)\ge0$;
  \item $(x+5)(x+6)\le0$;
  \item $(x+5)(x+6)>0$;
  \item $(x-7)(x+10)<0$;
  \item $(3x-1)(5-7x)\ge0$;
  \item $(10x+5)(3x+6)\le0$;
  \item $(9x+5)(54x+6)>0$;
  \item $(7-8x)(x+10)<0$;
\end{enumerate}
\end{multicols}
\end{question}

\begin{question}
  Решить обобщенным методом интервалов неравенства с произведением трех линейных многочленов:
\begin{enumerate}
  \item $(-x-3)(7-5x)(3+x)\ge0$;
  \item $(10x+100)(6x+3)x\le0$;
  \item $9x(30x+5)(-2x+0,5)>0$;
  \item $(0,8-8x)(0,09x+9)(1,5-2,5x)<0$;
\end{enumerate}
\end{question}

\begin{question}
  Решить обобщенным методом интервалов неравенства со степенными сомножителями:
\begin{enumerate}
  \item $(0,4x+0,2)^2(1-5x)\ge0$;
  \item $(22x-11)^3(33x+22)^2 x^3\le0$;
  \item $77x^4(2^5x+2^7)^3(3^3x-3)^2>0$;
  \item $(\sqrt{2}-2\sqrt{2}x)^5 (x/3+2/5)^2<0$;
\end{enumerate}
\end{question}

 \begin{question}
  Решить обобщенным методом интервалов неравенства с дробными выражениями:
  \begin{enumerate}[itemsep=6pt]
  \item $\frac{(x-2)}{(x+10)(x+8}\ge0$;
  \item $\frac{(x+2)(x-3)}{(10^2-10x)(x+8)}\le0$;
  \item $\frac{(2x-2)(5x-3)(25x+5)}{(1-x)(-x+9)}<0$;
  \item $\frac{(2x-2)(5x-3)(25x+5)}{(8-16x)(9+15x)(1000x+10000)}>0$;
  \item $\frac{(x-12)^3}{(x+80)^2(x+8)}\ge0$;
  \item $\frac{(3x+6)^3(1-2x)}{(10-0,01x)^4(x+8)}\le0$;
  \item $\frac{(3x-3)^5(8x-2)^6(75x+15)^3}{(1-x)^1(-x+9)^0}<0$;
\end{enumerate}
\end{question}
\end{multicols}
\end{exercises}
\end{document}


