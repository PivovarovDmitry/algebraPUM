%        File: 07.04.20.tex
%     Created: вс мар 29 07:00  2020 M
% Last Change: вс мар 29 07:00  2020 M
%
\documentclass[algebra,twocolumn]{pum}
\listnumber{5}
\date{07.04.20}
\classname{8-Д}
\lesson{9:30-11:10 }
\usepackage{icomma}
\begin{document}
\subsubsection*{Сравнение чисел}
Для сравнения двух действительных чисел необходимо из одного вычесть другое. Если результат оказывается положительным, значит уменьшаемое больше вычитаемого. Если результат оказывается отрицательным, значит уменьшаемое меньше вычитаемого. Если результат равен 0, то сравниваемые числа равны. Это правило основано на свойствах числовых неравенств и саму процедуру сравнения чисел превращает в вычисление алгебраических действий.

\subsubsection*{Свойства числовых неравенств}
\begin{enumerate}[nosep]
  \item Разрешено прибавлять любое число к обоим частям неравенства \\
    Следствие: любое слагаемое можно перенести из одной части неравенства в другу поменяв при этом знак на противоположный.
  \item Разрешено умножать обе части неравенства на одно и то же положительное число
  \item При умножении неравенства на -1 знак неравенства меняется на противоположный \\
    Следствие: можно умножать на любое отрицательное число, поменяв знак неравенства на противоположный.
\end{enumerate}

Однако чаще мы можем сказать какое из чисел больше, посмотрев на их запись в десятичной форме. Для этого мы сначала сравниваем их знаки, затем проверяем число разрядов и наконец в каждом разряде производим сравнение цифр, порядок которых известен заранее:
$$0<1<2<3<4<5<6<7<8<9$$
Если целые части оказываются равными, то процесс продолжается с разрядами после запятой.

Применяя это правило мы на самом деле пользуемся первым правилом, только в унифицированном виде. Оказывается любое действительное число можно представить с некоторой точностью в виде десятичной дроби и провести сравнение <<на глаз>>.

Но не всегда нужно представлять число в виде десятичной дроби. Можно воспользоваться простыми правилами сравнения рациональных и иррациональных чисел.

\begin{gather*}
  \begin{aligned}
    \frac{m_1}{n}>&\frac{m_2}{n},&\quad\text{если}\quad m_1>&m_2 \\
    \frac{m}{n_1}>&\frac{m}{n_2},&\quad\text{если}\quad n_1<&n_2 \\
    \sqrt{a}>&\sqrt{b},&\quad\text{если}\quad a>&b \\
    a^2>&b^2,&\quad\text{если}\quad a>&b
  \end{aligned}
\end{gather*}

\begin{exercises}
  \begin{question}
    Сравнить числа из $\mathbb{Z}$:
    \begin{multicols}{2}
      \begin{enumerate}[label=\arabic*)]
        \item 3 и 8
        \item 12 и 19
        \item 20 и 120
        \item 10 и -10
        \item -4 и 0
        \item 5 и 5
        \item $8\cdot9$ и $9\cdot8$ 
        \item $-5\cdot(6+7)$ и -75
        \item $10\cdot(4-9)$ и $100\cdot0,5$
        \item $20000000$ и $200000000$
      \end{enumerate}
    \end{multicols}
  \end{question}
  \begin{question}
    Сравнить числа из $\mathbb{Q}\setminus\mathbb{Z}$:
    \begin{multicols}{2}
      \begin{enumerate}[label=\arabic*)]
        \item $\frac{2}{10}$ и $\frac{4}{10}$
        \item $-\frac{15}{4}$ и $-\frac{16}{4}$
        \item $\frac{51}{100}$ и $\frac{255}{500}$
        \item $-4\frac{7}{8}$ и $-4\frac{3}{4}$
        \item $\frac{9}{4}$ и $\frac{9}{2}$
        \item $-\frac{14}{99}$ и $\frac{-24}{99}$
        \item $\frac{9}{2}$ и $\frac{18}{4}$
        \item $-7\frac{15}{40}$ и $-7\frac{32}{80}$
      \end{enumerate}
    \end{multicols}
  \end{question}
  \begin{question}
    Сравнить десятичные дроби:
    \begin{enumerate}[nosep]
        \item $0,208572$ и $-0,208572$
        \item $843,17934)$ и $843,17234$
        \item $-120,0000000056$ и $-120,000000056$
        \item $37,355(834)$ и $37,355(83)$
      \end{enumerate}
  \end{question}
  \begin{question}
    Сравнить числа из $\mathbb{R}$:
      \begin{enumerate}
        \item $\sqrt{0,9}$ и $\sqrt{0,4}$
        \item $3+\sqrt{37}$ и $4+\sqrt{26}$
        \item $\sqrt{8}+\sqrt{37}$ и $\sqrt{15}+\sqrt{26}$
        \item $\frac{8}{17}$ и $\frac{1}{2+\sqrt{17-12\sqrt{2}}}$
      \end{enumerate}
  \end{question}
\end{exercises}

\subsubsection*{Графическое представление чисел}

Каждому действительному числу можно сопоставить точку на числовой прямой. Для этого координата точки должна совпадать с числом. Сравнивать числа между собой можно сравнивая взаимное расположение точек на числовой прямой. Чем левее точка, тем меньше число. И наоборот, чем правее точка, тем больше число. Правда это справедливо, если числовая ось будет горизонтальной и направленной вправо. Расположить числа в порядке возрастания означает перечислить их слева направо как располгаются соотвествующие им точки числовой прямой. Если требуется расположить в порядке убывания, то перечислять нужно справа налево. Располагают же числа числа на прямой не точно, а относительно друг друга, т.е. попарно сравнивая между собой.

\begin{exercises}
\begin{question}
  Расположить в порядке убывания числа: $\sqrt{195},\sqrt{271},\sqrt{16},\sqrt{81}$
\end{question}
\begin{question}
  Расположить в порядке возрастания числа: $\sqrt{144},\sqrt{279},\sqrt{4},\sqrt{49}$
\end{question}
\end{exercises}

\begin{repeating}
  \begin{question}
    %\textcolor{darkcolortheme}{[748]} %Мерзляк
    В саду 60\% деревьев составляют вишни и сливы, из них 30\% составляют сливы. Какой процент всех деревьев сада соатвляют сливы?
  \end{question}
  \begin{question}
    %\textcolor{darkcolortheme}{[348]} %Мерзляк
    Если лыжник будет двигаться со скоростью 10 км/ч, то доберется в пункт назначения на 1 ча позже запланированного рвемени прибытия, а если будет двигаться со скоростью 15 км/ч -- то на 1 ч раньше. С какой скоростью он должен двигаться, чтобы прибыть в пункт назначения в запланированное время?
  \end{question}
  \begin{question}
    %\textcolor{darkcolortheme}{[304]} %Мерзляк
  Для откачивания воды из затопленного помещения были задействованы три насоса. Первый из них может выкачать всю воду за 12 ч, второй -- за 15 ч, а третий -- за 20 ч. Сначала в течение 3 ч работали первый и второй насосы, а затем подключили третий насос. За какое время была откачана вся вода.
  \end{question}
\end{repeating}

\end{document}
