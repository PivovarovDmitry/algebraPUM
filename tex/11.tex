%        File: 11.tex
%     Created: ср апр 15 10:00  2020 M
% Last Change: ср апр 15 10:00  2020 M
%
\documentclass[algebra,twocolumn]{pum}
\listnumber{11}
\date{16.04.20}
\classname{8-Д}
\lesson{11:30-13:20}
\begin{document}
\subsubsection*{Кубические неравенства}
Кубическое неравенство -- неравенство, в обеих частях которого стоят многочлены 3-ей степени. Простейшим примером кубического неравенства может быть неравенство вида: $$ax^3+bx^2+cx+d>0.$$

Существует утверждение, что любой многочлен 3-ей степени можно представить в виде: $$a(x-\alpha)(x-\beta)(x-\gamma).$$

Воспользуемся этим и составим таблицу возможных знаков каждого из слагаемых и их произведения:
\begin{table}[h]
  \centering
  \begin{tabular}{c|c|c||c}
    $x-\alpha$ & $x-\beta$ & $x-\gamma$ & произведение \\
    \hline
    $+$ & $+$ & $+$ & $+$ \\
    $+$ & $+$ & $-$ & $-$ \\
    $+$ & $-$ & $+$ & $-$ \\
    $+$ & $-$ & $-$ & $+$ \\
    $-$ & $+$ & $+$ & $-$ \\
    $-$ & $+$ & $-$ & $+$ \\
    $-$ & $-$ & $+$ & $+$ \\
    $-$ & $-$ & $-$ & $-$ \\
  \end{tabular}
\end{table}

Мы видим, что если нужно найти значения $x$, при которых неравенство положительно или отрицательно, то мы должно рассмотреть 4 варианта произведения. Вспоминая аналитический метод решения неравенств мы приходим к тому, что нужно рассмотреть 4 системы из трех неравенств, решить их, и записать ответ в виде объединения. Это потребоует многочисленных записей, в которых можно запутаться, и как следствие сделать ошибку.

Применить графический метод решения также нерационально, поскольку мы не знаем как выглядит график многочлена 3-ей степени, а если его строить по точкам, то границы интервалом можно найти лишь приближенно.

Точное решение можно получить воспользовавшись \emph{методом интервалов}.

\subsubsection*{Метод интервалов}
Метод интервалов основан на идее чередования интервалов, которые можно было обнаружить в графическом методе решения квадратных неравенств. Сравните два графика:
\begin{figure}[h]
  \centering
  \begin{tikzpicture}[scale=0.9]
    \begin{scope}
      \coordinate (A) at (-1,1);
      \coordinate (B) at (1,1);
      \coordinate (C) at (-2,1);
      \coordinate (D) at (2,1);
      \draw[thick,->] (C) -- node[above,yshift=3mm] {$-$} (D);
      \draw[domain=-1.41:1.41,smooth,ultra thick,darkcolortheme] plot ({\x},{\x*\x});
      \draw[ultra thick,darkcolortheme,fill=white] (A) circle (3pt);
      \draw[ultra thick,darkcolortheme,fill=white] (B) circle (3pt);
      \node[below left,yshift=-2mm] at (A) {$x_1$};
      \node[below right,yshift=-2mm] at (B) {$x_2$};
      \node[right] at (D) {$x$};
      \node[above,yshift=3mm] at (C) {$+$};
      \node[above,yshift=3mm] at (D) {$+$};
    \end{scope}
    \begin{scope}[shift={(5,0)}]
      \coordinate (A) at (-1,1);
      \coordinate (B) at (1,1);
      \coordinate (C) at (-2,1);
      \coordinate (D) at (2,1);
      \draw[thick,->] (C) -- node[above,yshift=3mm] {$+$} (D);
      \draw[domain=-1.41:1.41,smooth,ultra thick,darkcolortheme] plot ({\x},{-\x*\x+2});
      \draw[ultra thick,darkcolortheme,fill=white] (A) circle (3pt);
      \draw[ultra thick,darkcolortheme,fill=white] (B) circle (3pt);
      \node[below right,yshift=-2mm] at (A) {$x_1$};
      \node[below left,yshift=-2mm] at (B) {$x_2$};
      \node[right] at (D) {$x$};
      \node[above,yshift=3mm] at (C) {$-$};
      \node[above,yshift=3mm] at (D) {$-$};
    \end{scope}
  \end{tikzpicture}
\end{figure}

Здесь намеренно не нарисована ось ординат, потому что не играет роли, где именно относительно начала координат расположены графики. Видно, что если мы расставим на числовой прямой точки пересечения графика функции с осью абсцисс, то в этих точках происходит смена знака функции (произведения).

Аналогичная картина наблюдается и с прямой линией, которая может пересекать ось абсцисс только в одной точке (см. следующую страницу).
\begin{figure}[h]
  \centering
  \begin{tikzpicture}[scale=0.9]
    \begin{scope}
      \coordinate (A) at (-1,0);
      \coordinate (B) at (1,0);
      \coordinate (C) at (-2,0);
      \coordinate (D) at (2,0);
      \draw[thick,->] (C) -- (D) node[right] {$x$};
      \draw[domain=-1:1,smooth,ultra thick,darkcolortheme] plot ({\x},{\x});
      \draw[ultra thick,darkcolortheme,fill=white] (0,0) circle (3pt);
      \node[above,yshift=3mm] at (A) {$-$};
      \node[above,yshift=3mm] at (B) {$+$};
      \node[below,yshift=-2mm] at (0,0) {$x_1$};
    \end{scope}
    \begin{scope}[shift={(5,0)}]
      \coordinate (A) at (-1,0);
      \coordinate (B) at (1,0);
      \coordinate (C) at (-2,0);
      \coordinate (D) at (2,0);
      \draw[thick,->] (C) -- (D) node[right] {$x$};
      \draw[domain=-1:1,smooth,ultra thick,darkcolortheme] plot ({\x},{-1*\x});
      \draw[ultra thick,darkcolortheme,fill=white] (0,0) circle (3pt);
      \node[above,yshift=3mm] at (A) {$-$};
      \node[above,yshift=3mm] at (B) {$+$};
      \node[below,yshift=-2mm] at (0,0) {$x_1$};
    \end{scope}
  \end{tikzpicture}
\end{figure}

Поняв, что знаки чередуются, остается решить, с какого знака начинать. Обе левые картинки можно получить из правых отрожением относительно оси абсцисс, т.е. домножением функции на $-1$. Заметим, что старший коэффициент функций, которые изображены на правых рисунках отрицательный. Это и является условием для требуемого домножения.

\begin{pumbox}{Алгоритм}
  \begin{enumerate}[label=\arabic*.]
    \item Преобразовать левую часть кубического неравенства к виду $a(x-\alpha)(x-\beta)(x-\gamma)$.
    \item Разделить неравенство на $a$.
    \item Расставить числа $\alpha$, $\beta$ и $\gamma$ по возрастанию на числовой прямой, обозначив их кружками.
    \item Заштрифовать круги, если неравенство нестрогое.
    \item Расставить между отмеченными точками чередующиеся знаки справа налево, начиная с плюса.
    \item Записать объединение интервалов одного знака, который определяем из неравенства.
  \end{enumerate}
\end{pumbox}

\begin{exercises}
  \begin{question}
    Выполнить первый и второй шаг алгоритма для следующих неравенства и найти числа $\alpha$, $\beta$ и $\gamma$:
    \begin{enumerate}
      \item $-(x-2)(x+3)(x-5)>0$;
      \item $(1-x)(x+3)(8-x)>0$;
      \item $(x+13)(7-x)(8-x)<0$;
      \item $(x-2)(4-x)(x-5)<0$;
      \item $(10-x)(15-x)(16-x)\ge0$;
      \item $-(15-x)(5-x)(7-x)\ge0$;
      \item $(10-5x)(15-3x)(16-x)\le0$;
      \item $-(16-8x)(5-x)(7-x)\le0$;
    \end{enumerate}
  \end{question}
  Решить неравенства методом интервалов:
  \begin{question}
    \vspace{-13pt}
      \begin{enumerate}
        \item $(x-1)(x-3)(x-5)>0$;
        \item $(x+1)(x-1)(x-2)\ge0$;
        \item $(x-1)(x-2)(x+5)<0$;
        \item $(x+2)(x+1)(x-3)\le0$;
      \end{enumerate}
  \end{question}
  \begin{question}
    \vspace{-13pt}
      \begin{enumerate}
        \item $(2-x)(x+3)(x-7)<0$;
        \item $(5-x)(x-3)(x+12)>0$;
        \item $(3x-4)(1-x)(2x+1)<0$;
        \item $(2x-5)(7x+3)(x+8)<0$.
      \end{enumerate}
  \end{question}
  \begin{question}
    \vspace{-13pt}
      \begin{enumerate}
        \item $(x-3)(x^2-3x+2)\ge0$;
        \item $(2-x)(x^2-x-12)\le0$.
      \end{enumerate}
  \end{question}
  \begin{question}
    \vspace{-13pt}
    \begin{enumerate}
        \item $(2-4x)(x^2-x-2)<0$;
        \item $(-4-3x)(x^2+3x-4)>0$;
        \item $(3x-7)(x^2+2x+2)<0$;
        \item $(5x-8)(x^2-4x+5)>0$.
      \end{enumerate}
  \end{question}
\end{exercises}

\subsubsection*{Обобщенный метод интервалов}
Знак \textcolor{red}{не будет} чередоваться, если парабола пересекает ось абсцисс в одной точке.
\begin{figure}[h]
  \centering
  \begin{tikzpicture}[scale=0.9]
    \begin{scope}
      \coordinate (A) at (0,0);
      \coordinate (C) at (-2,0);
      \coordinate (D) at (2,0);
      \draw[thick,->] (C) node[above right,yshift=3mm] {$+$} -- (D) node[above left,yshift=3mm] {$+$} node[right] {$x$};
      \draw[domain=-1:1,smooth,ultra thick,darkcolortheme] plot ({\x},{\x*\x});
      \draw[ultra thick,darkcolortheme,fill=white] (A) circle (3pt);
      \node[below,yshift=-2mm] at (A) {$x_1$};
    \end{scope}
    \begin{scope}[shift={(5,0)}]
      \coordinate (A) at (0,0);
      \coordinate (C) at (-2,0);
      \coordinate (D) at (2,0);
      \draw[thick,->] (C) node[above right,yshift=3mm] {$-$} -- (D) node[above left,yshift=3mm] {$-$} node[right] {$x$};
      \draw[domain=-1:1,smooth,ultra thick,darkcolortheme] plot ({\x},{-\x*\x});
      \draw[ultra thick,darkcolortheme,fill=white] (A) circle (3pt);
      \node[below,yshift=-2mm] at (A) {$x_1$};
    \end{scope}
  \end{tikzpicture}
\end{figure}

Такая ситуация возникает в случае, если дискриминант неравенства равен нулю. В этом случае квадратный трехчлен можно записать в виде полного квадрата 
$$ax^2+cx+b=a(x-x_1)^2.$$
Это замечание обобщает метод интервалов на случай решения неравенств
$$(x-\alpha)^2(x-\beta)>0,\quad (x-\alpha)^3>0.$$
\begin{pumbox}{Измененные шаги в обобщенном методе}
  \begin{enumerate}
    \item[1*.] Преобразовать левую часть кубического неравенства к одному из следующих видов:
      \begin{gather*}
        a(x-\alpha)(x-\beta)(x-\gamma), \\
        a(x-\alpha)^2 (x-\beta), \\
        a(x-\alpha)^3.
      \end{gather*}
    \item[4*.] Расставить знаки справа налево, начиная с плюса, изменяя знак при прохождении точек $\beta$ и $\gamma$. При прохождении точки $\alpha$ изменяем знак, если $x-\alpha$ в нечетное степени, и оставляем его, если $x-\alpha$ в четной степени.
  \end{enumerate}
\end{pumbox}

\begin{exercises}
  Решить неравенства обобщенным методом интервалов:
  \begin{question}
    \vspace{-13pt}
    \begin{enumerate}
      \item $(x-2)^2(x-1)>0$;
      \item $(x+4)(x+3)^2<0$;
      \item $(x+2)(3-5x)^2<0$;
      \item $(3x-1)(x+1)^2>0$.
    \end{enumerate}
  \end{question}
  \begin{question}
    \vspace{-13pt}
    \begin{enumerate}
      \item $(8x-9)^3\ge0$;
      \item $(30-10x)^3\ge0$;
      \item $(0,2x-0,4)^3\le0$;
      \item $(1,5-4,5x)^3\le0$;
      \item $(\sqrt{2}x-\sqrt{8})^3>0$;
      \item $(2\frac{3}{4}-7\frac{5}{8}x)^3>0$;
      \item $(\sqrt{10}x-\sqrt{100})^3<0$;
      \item $(-\frac{1}{4}-\frac{15}{16}x)^3<0$.
    \end{enumerate}
  \end{question}
  \begin{question}
    \vspace{-13pt}
    \begin{enumerate}
      \item $(x+1)(x^2-x+x)>0$;
      \item $(3-x)(x^2+2x+1)^2<0$;
      \item $(4x^2+4x+4)(8-4x)^2<0$;
      \item $(3x^2+2x-1)(x+1)>0$.
    \end{enumerate}
  \end{question}
\end{exercises}
\end{document}
